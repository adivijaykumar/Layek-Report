% Chapter 1

\chapter{Fields Describing Free Particles} % Main chapter title

\label{Chapter1} % For referencing the chapter elsewhere, use \ref{Chapter1} 

\lhead{Chapter 1. \emph{Fields describing Free Particles}} % This is for the header on each page - perhaps a shortened title

%---------------------------------------------------%

\section{Introduction}

In the realm of classical physics, we first found out by Newton's formulation that the total amount of force acting on a body is directly proportional to its acceleration. We also realized that there is an equivalent formulation given by Lagrange, which involves a system-describing quantity called the Lagrangian, and a set of equations that map the trajectory of the body. We are also aware of the principles of quantum mechanics, and the wave mechanics, matrix mechanics models. Einstein also postulated his theory of relativity, which described classical physics at very high speeds.

The special theory of relativity postulated by Einstein and quantum mechanics are unified using a quantum theory to describe fields. Any Lagrangian here would have complete information about how the wavefunction of a particular system would eveolve in time, along with its probability densities.

\section{Vector Fields}
The electromagnetic field is a vector field. JC Maxwell gave the famous laws, which describe electric and magnetic fields. Given below are Maxwell's equations in free space

\begin{subequations}
\begin{align}
{\nabla}.E=\rho,\\
{\nabla}.B=0\\
\frac{\partial B}{\partial t}&=-\nabla \times E,\\
\frac{\partial B}{\partial t}&=\nabla \times B - 4\pi j,
\end{align}
\end{subequations}

For these electric and magnetic fields, we can define potentials as follows
\begin{subequations}
\begin{align}
E={-\nabla}E+\frac{\partial A}{\partial t}\\
B=\nabla \times A
\end{align}
\end{subequations}

If we write $A^{\mu} = (A^0,\overset{\xrightarrow{}}{A})$, we could write
\begin{equation}
F_{\mu\nu} = \partial_\mu{A_\nu}-\partial_\nu{A_\mu_}
\end{equation}

where 
 is a tensor, called very appropriately as the field strength tensor.


Using this, the Lagrangian for the electromagnetic field can be constructed as follows
\begin{equation}
{L} = -\frac{1}{4}F_{{\mu}\nu}F^{{\mu}\nu}
\end{equation}

\section{Particles with Spin Zero}
The Lagrangian for a scalar field is as follows:

\begin{equation}
{L} = \frac{1}{2}({\partial^{\mu}} \Phi)({\partial_{\mu}} \Phi^*) - \frac{1}{2}m\Phi\Phi^*
\end{equation}

This field is called the Klein-Gordon field, and describes spin zero particles. On applying the Euler Lagrange equations using this Lagrangian, we can arrive at equations of motion for this field.

\begin{equation}
(\partial^{\mu}\partial_{\mu}+m^2)\Phi(x)=0
\end{equation}

\section{Particles with spin-$\frac{1}{2}$}

British physicist Paul Dirac in 1928 derived a relativistic wave equation in 1928, which described all spin-$\frac{1}{2}$ particles. This equation, now known famously as the Dirac equation, is as follows

\begin{equation}
(i\gamma^{\mu}\partial_\mu-m)\psi(x)=0
\end{equation}
where
\begin{eqnarray}
\gamma^0=\begin{bmatrix}
1 & 0\\
0 & 1
\end{bmatrix}\\
\gamma^i=\begin{bmatrix}
0 &  \sigma_i\\
-\sigma_i & 0
\end{bmatrix}
\end{eqnarray}

The Klein-Gordon equation offers only one solution. The Dirac equation on the other hand gives four solutions, which can be written in matrix form as 


\begin{eqnarray}
\psi_a=\begin{bmatrix}
1\\
0\\
p_z/(p^0+m)\\
(p_x+ip_y)/(p^0+m)
\end{bmatrix}
\psi_b=\begin{bmatrix}
0\\
1\\
(p_x-ip_y)/(p^0+m)\\
-p_z/(p^0+m)\\
\end{bmatrix}\\
\psi_c=\begin{bmatrix}
p_z/(p^0+m)\\
(p_x+ip_y)/(p^0+m)\\
1\\
0
\end{bmatrix}
\psi_d=\begin{bmatrix}
(p_x-ip_y)/(p^0+m)\\
p_z/(p^0+m)\\
0\\
1
\end{bmatrix}
\end{eqnarray}

The Dirac Field Lagrangian corresponding to the Dirac equation is given by

\begin{equation}
{L} = \overset{-}{\psi}(i{\gamma^\mu}\partial_\mu-m)\psi
\end{equation}


