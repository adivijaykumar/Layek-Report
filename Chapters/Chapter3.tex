% Chapter Template

\chapter{Non-Abelian Gauge Theory} % Main chapter title

\label{Chapter3} % Change X to a consecutive number; for referencing this chapter elsewhere, use \ref{ChapterX}

\lhead{Chapter 3. \emph{Non-Abelian Gauge Theory}} % Change X to a consecutive number; this is for the header on each page - perhaps a shortened title

%----------------------------------------------------------------------------------------
%	SECTION 1
%----------------------------------------------------------------------------------------

\section{Yang-Mills Theory of non-Abelian gauge fields}

As we have already seen, a Lagrangian should remain invariant under global/local transformations. This means that the Lagrangian should have both global and local symmetries.

For spin-1 fields, like the electromagnetic field, U(1) symmetries exist. For other fields, such as the gluonic field for instance, SU(N) symmetry exists.

Let's consider one such system given by 


Let $\psi^{'} = U\psi$, $U \in SU(N)$. We can then say

\begin{equation}
L_{\psi^{'}} = \overset{_}{\psi}U^{-1}(i{\gamma^\mu}\partial_\mu-m)U\psi
\end{equation}
\begin{equation}
L_{\psi^{'}} = L_\psi + \overset{_}{\psi}U^{-1}(i{\gamma^\mu})\partial_\mu(U)\psi \\
\end{equation}
This hence warants the introduction of a new gauge field such that $L_{\psi^{'}} = L_\psi$

We introduce covariant derivative $D_{\mu}$, such that the modified Lagrangian is given by
\begin{equation}
{L} = \overset{_}{\psi}(i{\gamma^\mu}D_\mu-m)\psi
\end{equation}
$D_\mu = \partial_\mu +igT_{a}A^{a}_{\mu}$, where $g$ is coupling constant, $T_{a}$ are the generators of SU(N), and $A^{a}_{\mu}$ is the gauge field.

It can be proved that the this Lagrangian is invariant under SU(N), and the following condition is satisfied

\begin{equation}
T_{a}{A^{a}_{\mu}}^{'} = \frac{i}{g}(\partial_{\mu}U)U^{-1} + UT_{a}A^{a}_{\mu}U^{-1}
\end{equation}
